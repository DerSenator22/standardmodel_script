% commands
\newcommand*{\name}[1]{\textsc{#1}}  % names of persons
\newcommand*{\definition}[1]{\textbf{#1}}  % new defined “words”

\renewcommand*{\dh}{d.\,h. } 
\newcommand*{\zB}{z.\,B. }
\newcommand*{\ZB}{Z.\,B. }
\newcommand*{\Dh}{D.\,h. }

% the following yields a cell in a table, where you need multiple lines, 1st
% argument optional, default is "c"
\newcommand{\lmultlinecell}[2][c]{\begin{tabular}[#1]{@{}l@{}}#2\end{tabular}}
\newcommand{\cmultlinecell}[2][c]{\begin{tabular}[#1]{@{}c@{}}#2\end{tabular}}
\newcommand{\rmultlinecell}[2][c]{\begin{tabular}[#1]{@{}r@{}}#2\end{tabular}}

% the following yields a nicer line over symbols, e.g. for setting the
% covariant adjungated
% use: \xoverline[width percent]{symb}, optdefault is 0.75 
\makeatletter
\newsavebox\myboxA
\newsavebox\myboxB
\newlength\mylenA

\newcommand*\xoverline[2][0.75]{%
	\sbox{\myboxA}{$\m@th#2$}%
	\setbox\myboxB\null% Phantom box
	\ht\myboxB=\ht\myboxA%
	\dp\myboxB=\dp\myboxA%
	\wd\myboxB=#1\wd\myboxA% Scale phantom
	\sbox\myboxB{$\m@th\overline{\copy\myboxB}$}%  Overlined phantom
	\setlength\mylenA{\the\wd\myboxA}%   calc width diff
	\addtolength\mylenA{-\the\wd\myboxB}%
	\ifdim\wd\myboxB<\wd\myboxA%
	\rlap{\hskip 0.5\mylenA\usebox\myboxB}{\usebox\myboxA}%
	\else
	\hskip -0.5\mylenA\rlap{\usebox\myboxA}{\hskip 0.5\mylenA\usebox\myboxB}%
	\fi}
\makeatother
