\chapter{Luminosität und Wirkungsquerschnitt im Experiment}

Die Anzahl von gestreuten Teilchen in einem Raumwinkelelement ergibt sich zu:
\[ dN = N d\Omega = \underbrace{\int d x_1^2 N_1 \mu_1 (\vec{x_1}) N_2 
(\vec{x_2})}_{\mathit{integrierte\ Luminosität}} d\sigma \]
Für Teilchenstrahlen mit Teilchenzahlen $N_1, N_2$ und Verteilungsdichten 
$\mu_1, \mu_2$. \\
\\
\textbf{Bild}\\
\\

\section{Luminosität}

Bei Kreisbeschleunigern wie dem LHC oder dem LEP laufen Teilchen in $n_b$ 
Paketen, sogenannten bunches, im Kreis. \\
Dabei werden Umlauffrequenzen von $f = 11245,5\,Hz$ erreicht. $\left( f \approx 
\frac{c}{26,7\,km} \right) $ \\
Wir betrachten nun die instantane Luminosität: 
\[ \mathcal{L} (t) = n_b f \tilde{N_1} \tilde{N_2} \int dx^2 \mu_1(\vec{x_1}) 
\mu_2(\vec{x_2})  \]
Mit $ \tilde{N_1},\ \tilde{N_2}$ als die Anzahl der Teilchen pro Paket. \\
In guter Näherung sind $\mu_1,\ \mu_2$ unabhängig in x- und y-Richtung, d.h. 
man kann sie faktorisieren. 
\begin{align*}
\int d^2x \mu_1(\vec{x_1}) \mu_2(\vec{x_2}) &= \int dx\, dy \mu_1(\vec{x}) 
\mu_1(\vec{y}) \mu_2(\vec{x}) \mu_2(\vec{y}) \\
&= \underbrace{\int dx \mu_1(\vec{x}) \mu_2(\vec{x})}_{\Omega_x} 
\underbrace{\int \mu_1(\vec{y}) \mu_2(\vec{y})}_{\Omega_y}
\end{align*}  
Dabei sind $\Omega_x$ und $\Omega_y$ die Überlappintegrale.\\
Die $\mu_i$ sind normiert:
\[ \int \mu_i (x) dx = 1 \hspace{0.5cm} \Longrightarrow \hspace{0.5cm} 
\mathcal{L}(t) = n_b f \tilde{N_1} \tilde{N_2} \Omega_x \Omega_y \]

\section{Van der Meer-Scan}

Der Van der Meer-Scan ist eine Methode zur Bestimmung der Luminosität, die von 
S. van der Meer entwickelt wurde. \\
Dafür wird die relative Luminosität bzw. Wechselwirkungsrate für um b$_x$, 
b$_y$ versetzte Teilchenstrahlen gemessen:
\[ \mathcal{L}(b_x) = n_b f \tilde{N_1} \tilde{N_2} \int dx \mu_1 (x) \mu_2 
(x-b) \cdot \Omega_y \]
Gleiches gilt analog für b$_y$. \\
\[ \int_{-\infty}^{\infty} db_x \mathcal{L} (b_x) = n_b f \tilde{N_1} 
\tilde{N_2} \Omega_y \underbrace{\int_{-\infty}^{\infty} db_x 
\int_{-\infty}^{\infty} \mu_1 (x) \mu_2 \underbrace{(x-b_x)}_{b'} 
dx}_{\underbrace{\int_{-\infty}^{\infty} db' \int_{-\infty}^{\infty} dx 
\mu_1(x) \mu_2(b')}_{1}} \]
Daraus folgt: 
\[ \int_{-\infty}^{\infty} db_x \mathcal{L}(b_x) = n_b f \tilde{N_1} 
\tilde{N_2} \Omega_y \]
Die linke Seite der Gleichung ist als Integral der Wechselwirkungsrate messbar. 
Die rechte wird mit dem Strahlstrom gemessen. \\
Hieraus lässt sich dann $\Omega_y$ und analog auch $\Omega_x$ bestimmen. \\
Dabei muss $\mathcal{L}(t)$ nicht absolut bestimmt werden, sondern nur relativ. 
\[ \Omega_x = \frac{\mathcal{L} (b_x = 0)}{\int db_x \mathcal{L}(b_x)} \]
\\
Wir definieren nun $\Sigma_{x,y}$:
\[ \Sigma_{x,y} = \frac{1}{\sqrt{2 \pi}} \frac{1}{\Omega_{x,y}} \]
dann:
\[ \mathcal{L} (t) = \frac{n_b f \tilde{N_1} \tilde{N_2}}{2 \pi \Sigma_x 
\Sigma_y} \]
Für gaußförmige Strahlprofile ist
\[ \mathcal{L} (b_x) = c \cdot \mathrm{e}^{- \frac{b_x^2}{2 \sigma_x^2}} \] 
Daraus folgt: 
\[ \Sigma_x = \frac{1}{\sqrt{2 \pi}} \frac{\int_{\infty}^{\infty}db 
\mathcal{L}(b)}{\mathcal{L}(0)} = \sigma_x \]
\textbf{Beispiel LHC:} 
\[ \Sigma_{x,y} \approx 60\,\mu m \]
beim Scan: 
\begin{align*}
\tilde{N}_{1/2} &\approx 10^{10}\, \mathit{Protonen} \\
n_b &\approx 1
\end{align*}
Für die Luminosität beim Scan gilt dann:
\[ \mathcal{L}_{Scan} (t) = \frac{f n_b \tilde{N_1} \tilde{N_2}}{2 \pi (60\,\mu 
m )^2} = 4,9 \cdot 10^{29}\,cm^{-2} s^{-1} \]
Nominell beim LHC:
\[ \tilde{N_{1/2}} = 10^{11},\ n_b \approx 2800 \]
\[ \Longrightarrow \mathcal{L}_{LHC} (t) \approx 10^{34} cm^{-2} s^{-1} \]
Daraus ergibt sich eine totale Luminosität pro Jahr von:
\[ 1\,a \approx \pi \cdot 10^7\,s \]
Die Betriebszeit beträgt:
\[ t \approx \frac{1\,a}{\pi} = 10^7\,s \]
\[ \mathcal{L}_{int} \approx 10^{41}\,cm^{-2} \]
\\
Die Einheit des Wirkungsquerschnittes ist:
\[ [\sigma] = 1\,\mathit{barn} = 10^{-24}\,cm^{2} = 100\, fm^2 \]
Für seltene Prozesse mit $\sigma = 1\,fb$ beträgt die Anzahl an Ereignissen:
\[ N = \mathcal{L}_{int} \cdot \sigma \approx 100 \]
Bei einer Genauigkeit von:
\[ \frac{\Delta \mathcal{L}}{\mathcal{L}} \approx 2-5\% \]

\section{Luminositätsmessung durch Referenzwirkungsquerschnitt}

\[ N = \sigma \cdot \mathcal{L}_{int}  \]
N ist die Anzahl der Ereignisse, $\mathcal{L}_{int}$ ist die gesuchte Größe und 
$\sigma$ ergibt sich aus der Theorie, wie zum Beispiel der 
Quantenelektordynamik. \\
Im Detektor gilt:
\[ N_{det} = \sigma \mathcal{L}_{int} \cdot \epsilon + \sigma_u 
\mathcal{L}_{int} \cdot \epsilon_u  \]
Mit $\sigma_u$ als Wirkungsquerschnitt des Untergrundes. Die Größen $\epsilon$ 
und $\epsilon_u$ beschreiben die Detektoreffizienz, bzw. -akzeptanz und sind 
kleiner 1. \\
\textbf{Beispiel LEP:} Dabei handelt es sich um einen $e^+ e^-$-Beschleuniger. 
\\
Als Referenz wird die Bhabha-Streuung genutzt. $e^+e^- \longrightarrow e^+e^-$\\
\\
\textbf{Bild}\\
\\
\[ S = (p_1 + p_2)^2 = (p_3 + p_4)^2 \]
\[ t = (p_3 - p_1)^2 = (p_4 - p_2)^2 = - \frac{s}{2} (1 -\cos{\theta}) \]
für masselose Teilchen. 
\[\frac{d\sigma}{d\Omega} = \frac{1}{4s} \left( \frac{3 + \cos{\theta}^2}{1 - 
\cos{\theta}} \right)^2 \alpha^2 \]
mit der elektromagnetischen Kopplung:
\[ \alpha = \frac{\mathrm{e}^2}{4\pi} \]
Für kleine Streuwinkel ($\sigma$ groß, $\theta << 1$) 
\[ \cos{\theta} \approx 1 - \frac{\theta^2}{2} \]
\[ d\Omega = d\phi d \cos{\theta} = 2\pi \theta d\theta \hspace{0.5cm} 
\Longrightarrow \hspace{0.5cm} \frac{d\sigma}{d\theta} = 
\frac{d\sigma}{d\Omega} 2\pi \theta = \frac{32\pi \alpha^2}{s} 
\frac{1}{\theta^3} \sim \frac{1}{\theta^3} \]
\\
\textbf{Bild}\\
\\
Messgenauigkeit der Luminosität ist begrenzt durch Theorievorhersage für 
$e^+e^- \longrightarrow e^+e^-$:
\[ \frac{\Delta \mathcal{L}}{\mathcal{L}} \approx 0,05\% \]
\textbf{Beispiel LHC:} Luminositätskalibration mit elastischer pp-Streuung. \\
\begin{itemize}
	\item[] elastische Streuung: pp $\longrightarrow$ pp
	\item[] unelastische Streuung: pp $\longrightarrow$ Hadronen
\end{itemize}
Dabei ist der Referenzprozess pp $\longrightarrow$ pp einigermaßen gut 
berechenbar.\\
\\
\textbf{Bild}\\
\\
Der totaler Wirkungsquerschnitt ergibt sich aus:
\[ \sigma_{tot} = \sigma_{el} + \sigma_{unel} \] 
Der Wirkungsquerschnitt für die elastische Streuung lässt sich nur mit 
spezieller Beschleunigerkonfiguration messen. \\
Der Wirkungsquerschnitt der unelastischen Streuung ist als Rate der 
Hadronenproduktion messbar. \\
Die Genauigkeit der Methode beträgt etwa $2-3\%$
