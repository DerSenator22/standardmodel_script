\chapter{Einführung}
\section{Ziel der Vorlesung}
Das Ziel der Vorlesung wird es sein, die theoretischen und experimentellen
Grundlagen der Teilchenphysik zu betrachten. Wir werden Streureaktionen an 
Beschleunigern untersuchen und Lagrangedichten und Symmetrien des 
Standardmodells aufstellen. Dies wird uns helfen, einfache Vorhersagen des 
Modells zu machen und diese experimentell zu überprüfen

\section{Grundlagen}
Wir wollen nun kurz einen Einblick in den aktuelle Stand des Standardmodells
geben und einige Grundbegriffe wiederholen.

\subsection{Heutiger Stand}
Das Standarmodell besteht aus fundamentalen Freiheitsgraden. Diese sind heute
als \definition{Elementarteilchen} bekannt. Dazu zählen die
\definition{Leptonen}:
%
\begin{equation*}
	\begin{pmatrix}
		\nu_e \\\c = 1
		e
	\end{pmatrix}
	%
	\begin{pmatrix}
		\nu_{\mu} \\
		\mu
	\end{pmatrix}
	%
	\begin{pmatrix}
		\nu_{\tau} \\
		\tau
	\end{pmatrix}
\end{equation*}
%
und die \definition{Quarks}: 
\begin{equation*}
	\begin{pmatrix}
		d \\
		u
	\end{pmatrix}
	%
	\begin{pmatrix}
		c \\
		s
	\end{pmatrix}
	%
	\begin{pmatrix}
		t \\
		b
	\end{pmatrix}
\end{equation*}
%
Sowie die sogenannten \definition{Eichbosonen}: $ \gamma^{\mu}$ (Photon),
$Z^{\mu}$ (Z-Boson), $W^{\pm \mu}$ (W-Bosonen) und $g^{\mu}$ (Gluon).
Zu dieser Liste gehört auch das \name{Higgs}-Boson, für das es 2013 den
Nobelpreis gab.
Das Standardmodell (SM) erklärt die Eigenschaften dieser Teilchen (sowie 
zusammengesetzter Teilchen (Protonen, Neutronen, $ \dots $, Atomkerne, Atome, 
Moleküle, $ \dots $).
Für das Verständnis wird die Vereinigung der Quantenmechanik sowie der 
Relativitätstheorie benötigt. 
Dies führt uns dann zur Quantenfeldtheorie.

\section{Wiederholung der speziellen Relativitätstheorie}

In der Vorlesung sowie in der Kern- und Teilchenphysik wird hauptsächlich mit 
den sogenannten natürlichen Einheiten gearbeitet. Dies bedeutet, dass 
einige wichtigen Größen wie das Plancksche Wirkungsquantum und die 
Lichtgeschwindigkeit auf 1 gesetzt werden: $ \hbar = c = 1 $.
Daraus folgen folgende Einheiten:
%
\begin{equation*}
	\left[ E \right] = \left[ m \right] = \left[ t^{-1} \right] = \left[ 
	l^{-1} \right] = \SI{1}{\eV}
\end{equation*}
% 
Die Umrechnung ergibt sich durch Multiplikation mit den auf 1 gesetzten
Größen. Zum Beispiel:
%
\begin{equation*}
	[ t ] = \SI{1}{\per \eV} \rightarrow \frac{\left( \SI{1}{\per \eV} 
	\right)}{\hbar} = \SI{6,6e-16}{\s} \\ 
	\left[ l \right] = \SI{1}{\per \eV} \rightarrow \frac{\left( \SI{1}{\per 
	\eV} 
	 \right)}{\hbar c} = \SI{1,97e-7}{\m}
\end{equation*}
%
Die Lichtgeschwindigkeit (c=1) ist dabei in allen Bezugssystemen
(Inertialsystem) gleich.
Für die Berechnungen sind außerdem die \definition{4-Vektoren} wichtig. 
%
\begin{equation*}
	x^{\mu} = \left( x^0 , x^i \right) = \left( t, \vec{x} \right) 
\end{equation*}
%
Dies ist der \definition{4-Ortsvektor}.
Für den \definition{4-Impuls} erhalten wir:
%
\begin{equation*}
	p^{\mu} = \left( p^0 , p^i \right) = \left( E, \vec{p} \right) 
\end{equation*}
%
Die \definition{Ableitung} schreiben wir:
%
\begin{equation*}
	\partial_{\mu} = \frac{\partial}{\partial{x^{\mu}}}
\end{equation*}
%
4-Vektoren mit Index oben nennen wir \definition{Kontravariant}. Diese sind in
im vektoriellen Anteil, also in den letzten drei Komponenten positiv. Mit 
Index unten werden \definition{kovariante} Vektoren bezeichnet. Bei diesem
sind die unteren Einträge negativ. 
Um die beiden Typen von Vektoren ineinander umzuwandeln benötigt es den 
\definition{Metrischen Tensor}. 
%
\begin{equation*}
	g_{\mu \nu} = g^{\mu \nu} = 
	\begin{pmatrix} 
		1 & 0 \\
		0 & \mathbb{1}_{3x3}
	\end{pmatrix}
\end{equation*}
%
Die Umrechnung erfolgt dann wie folgt:
%
\begin{equation*}
	 a_{\mu} := g_{\mu \nu} a^{\nu} \\
	 a^{\nu} = g^{\mu \nu} a_{\nu}
\end{equation*}
%
Die Berechnung von Skalarprodukten lässt sich über den metrischen Tensor
herleiten:
%
\begin{equation*}
	x \cdot g = x^{\mu} g_{\mu} = x^{\mu} y^{\nu} g_{\mu \nu} = x_{\mu} y_{\nu}
	g^{\mu \nu} = x^0 y^0 - \vec{x} \vec{y}
\end{equation*}
%
Zum Beispiel: 
%
\begin{equation*}
	\partial_{\mu}(x,p) = \frac{\partial}{\partial{x^{\mu}}} x^{\nu} p_{\nu} = 
	\delta_{\mu}^{\nu} p_{\nu} = p_{\mu}
\end{equation*}
%
Für die relativistische Rechnung ist außerdem die
\definition{Lorentztransformation} wichtig.
Die Matrix $\Lambda^{\mu}_{\nu}$ entspricht der Lorentztransformation und ist 
definiert durch:
%
\begin{equation*}
	\left(\Lambda_x \right)^{\mu} \left(\Lambda_y \right)^{\nu} g_{\mu \nu}
	= x^{\mu} y^{\nu} g_{\mu\nu}
\end{equation*}
%
Wir betrachten nun das Beispiel eines Boosts:
%
\begin{equation*}
	\Lambda^{\mu}_{\nu} = 
	\begin{pmatrix}
		\gamma & \beta_\gamma & 0 & 0 \\
		\beta_\gamma & \gamma & 0 & 0 \\
		0 & 0 & 1 & 1 \\
		0 & 0 & 1 & 1
	\end{pmatrix}
\end{equation*}
% 
mit 
%
\begin{equation*}
	\beta = \frac{v}{c} \\
	\gamma = \frac{1}{\sqrt{1-\beta^2}}
\end{equation*}
%
Des Weiteren betrachten wir einen Lichtstrahl von $\left( 0, \vec{0} \right)$ 
nach $\left(t, \vec{x} \right)$.
Da gilt: $ c = 1 $  folgt, dass $ \left| \vec{x} \right| = t $.
Daraus folgt, dass in jedem Inertialsystem gilt: 
%
\begin{equation*}
	 t^2 = {\left| \vec{x} 
	 	\right|}^2
\end{equation*}
%
Für die Elektrodynamik betrachten wir die kovariante \name{Maxwell}-Gleichung
mit dem \definition{4-Potential} $ A^{\mu}(x) = \left( \varphi (x),
\vec{A}(x)\right) $ und der \definition{4-Stromdichte} $ j^{\mu}(x) = \left( 
\varrho (x), \vec{j}(x)\right) $. Der \definition{Feldstärketensor} ergibt sich
dann wie folgt:
%
\begin{equation*}
	F^{\mu \nu} = \partial^{\mu} A^{\nu}(x) - \partial^{\nu} A^{\mu}(x)
\end{equation*}
%
Die \definition{\name{Maxwell}-Gleichung} hat dann folgende Form:
\begin{equation*}
	\partial_{\mu} F^{\mu \nu} = j^{\nu}
\end{equation*}
%
Der \definition{\name{Lagrange}-Formalismus} ist vor allem für relativistische
Theorien wichtig. Die \definition{\name{Lagrange}-Funktion} lässt sich über
die \definition{\name{Lagrange}dichte} berechnen:
%
\begin{equation*}
	L(t) = \int d^3x \mathcal{L} (x)
\end{equation*}
%
Das besondere an der \name{Lagrange}dichte ist, dass sie lorentzinvariant ist,
sich also unter der Lorentztransformation nicht verändert. Aus ihr erhalten wir
des Weiteren die Bewegungsgleichung. Eine Forderung an Theorien ist, dass 
diese lorentzinvariant sind.
Am Beispiel der Elektrodynamik:
%
\begin{equation*}
	\mathcal{L}(x) = \frac{1}{4} F_{\mu \nu} F^{\mu \nu} - j_{\mu} A^{\mu}
\end{equation*}
%
Mit Hilfe der Euler-Lagrange-Gleichung erhalten wir die kovariante 
\name{Maxwell}-Gleichung.

\section{Wiederholung der Quantenmechanik}
Wir betrachten zu erst allgemeine Postulate der Quantenmechanik.
Die Quantenmechanik wird mit Hilfe von Zuständen beschrieben. Diese sind 
Vektoren der Hilberträume. Zu jeder Observablen gehört außerdem ein hermitescher
Operator und die Eigenwerte der Operatoren sind die Messwerte dieser
Observablen.
Zum Beispiel:
%
\begin{equation*}
	\ket{\psi} = \sum_i{\scap{a_i}{\psi} \ket{a_i}}
\end{equation*} 
%
Dabei sind $ a_i $ die Eigenwerte und $ \scap{a_i}{\psi} $ die
Wahrscheinlichkeitsamplitude.
$ P_{\psi_i a_i} = {\left| \scap{a_i}{\psi} \right|}^2 $ ist die 
Wahrscheinlichkeit, den Zustand $\psi$ im Zustand $a_i$ zu finden.
Das System wird beschrieben durch $ \exlara{\Phi}{\mathcal{O}}{\Psi} $. 
Dabei sind $\ket{\phi}$ und $\ket{\psi}$ die Zustände. 
Für die Zeitentwicklung gilt: 
%
\begin{equation*}
	\i \frac{\d}{\d t} \exlara{\phi}{\mathcal{O}}{\psi} = 
	\exlara{\phi}{\underbrace{\left[\mathcal{O},H\right]}_{\textit{Kommutator}}}
	{\psi}
\end{equation*} 
%
falls gilt : 
%
\begin{equation*}
	\frac{\partial}{\partial t} \mathcal{O} = 0 
\end{equation*}  
%
Physikalisch interessant ist vor allem $ \Phi = \Psi $. Dann erhalten wir: 
$ \left\langle \mathcal{O} \right\rangle = \exlara{\psi}{\mathcal{O}}{\psi} $
Also den Erwartungswert. Die Normierung ist $ \scap{\Psi}{\Psi}$.
$\scap{\phi}{\psi}$ stellt die Wahrscheinlichkeit des Übergangs von 
$\ket{\psi}$ nach $\ket{\phi}$ dar.
Die Quantenmechanik ist des Weiteren eine Einteilchentheorie, es gilt also:
%
\begin{equation*}
	\left[ x,p \right] = \i \hbar = \i
\end{equation*}
%
Die Vielteilchentheorie beschäftigt sich unter anderem mit gemischten 
Zuständen. Dafür wird der \definition{Dichteoperator} verwendet.
Zum Beispiel: 
%
\begin{equation*}
	\varrho = \sum_i{W_i \ket{\psi_i} \bra{\psi_i}}
\end{equation*} 
%
Wenn es sich um diskrete Zustände handelt. Für kontinuierliche erhält man:
%
\begin{equation*}
	\varrho = \int{dp\ g(p) \ket{p} \bra{p}}
\end{equation*}
%
Dabei geben $W_i$ und $g(p) $ die Wahrscheinlichkeit an,