\chapter{Lagrangedichte des Standardmodells}

Bisherige Theorie:
\[ \text{Dynamik \{Hamiltonoperator, Struktur der Wechselwirkung\}} \rightarrow 
T_{fi} \rightarrow \sigma \]
Die Wechselwirkung wird durch 
\[ (\Dalembert + m^2) V = …  \] beschrieben. \\
Daraus folgt $T_{fi}$, welches proportional zum Impuls $q$ ist, was wiederum zu 
$\nicefrac{1}{q^2 - m^2}$. Dies gilt für klassische Streuung. \\
\\
Eine Möglichkeit besteht darin, die Resonanz zu betrachten, also:
\[ q^2 = m^2 \]
Daraus folgt eine einfache Breite $\Gamma$. \\
Dies wird experimentell beobachtet und ist mit den Teilchentypen identifiziert. 
\\
\textbf{hien scheint was zu fehlen} \\
Wir erhalten so den Teilcheninhalt des Standardmodells. \\
\begin{outline}
	\1 Spin-$\nicefrac{1}{2}$-Fermionen 
	\1 Leptonen
		\2 $e_R: \left(\begin{array}{c} \nu_e \\ e_l \end{array}\right) $
		\2 $\mu_R: \left(\begin{array}{c} \nu_{\mu} \\ \mu_l \end{array}\right) 
		$
		\2 $ \tau_R: \left(\begin{array}{c} \nu_{\tau} \\ \tau_l 
		\end{array}\right) $
	\1 Quarks
		\2 $ \left(\begin{array}{c} u_l \\ d_l \end{array}\right) $
		\2 $ \left(\begin{array}{c} c_l \\ s_l \end{array}\right) $
		\2 $ \left(\begin{array}{c} t_l \\ b_l \end{array}\right) $
\end{outline}
Experimentell: keine $V_r$ \\
\newline
Spin 1 Bosonen: 
\begin{center}
	\begin{tabular}{|c|c|c|c|}
		\hline Photonen: & $A^{\mu}$ & $ W^{\pm \mu} $ & $ Z^{\mu}$ \\ 
		\hline Masse: & 0 & $m_{W \neq 0} $ & $ m_{z \neq 0} $ \\ 
		\hline 
	\end{tabular} 
\end{center}
Des Weiteren existieren Gluonen deren Masse 0 ist und die durch die QCD 
beschrieben werden. $ G_a^{\mu}$ mit $ a=1,…,8 $. \\
\newline
Spin 0 Bosonen:
Unter diese Kategorie fällt das Higgs-Teilchen, dessen Existenz experimentell 
noch nicht 100\% nachgewiesen ist ($\ge 99\%$). \\
\newline
Das Ziel: Die Dynamik der Teilchen verstehen und beschreiben. \\
$\rightarrow$ Angebe des entsprechenden Theoriemodells \\
$\rightarrow$ speziell des Hamiltonoperators \\
$\rightarrow$ bzw. Beschreibung über Lagrangefunktion \\
\newline 
Das Ergebnis ist, dass das Standardmodell eine Eichtheorie ist. Die Eichgruppe 
ist elektroschwach $SU_l (2) \times U(1)$ \\
Durch das l wird zwischen links- und rechtshändigen Teilchen ($ \hat{=} $ Spin) 
$ \Rightarrow $ experimentelle Resultate \\
$\rightarrow$ massive Eichbosonen $W^{\pm}, Z$ sowie masselose Eichbosonen 
$A/\gamma$ (Photonen) \\
Die Masse der Teilchen wird durch den Higgsmechanismus generiert. Es existiert 
also ein skalares Feld. \\
Die Parameter der Theorie sind die 2 Eichkopplungen.  
