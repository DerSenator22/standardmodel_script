\chapter{Theoretische Grundlagen von Streureaktionen}

Ziel: Bild \\
\begin{itemize}
\item $1+2 \to 1' + ... + n'$ über Wechselwirkung (Ww). 
\item Quantenmechanische Beschreibung 
\item Definition des Wirkungsquerschnitts (Wq)
\end{itemize}

\section{Impulsoperator, Teilchenzustände}

Translationsinvarianz $\rightarrow$ erhaltene Größe "'Impuls" $\rightarrow$ 
Impulsoperator $\hat p$. \\
Die Translation wird durch erzeugt durch:
\[ T(a) = \mathrm e^{-ia \hat p} \]
d.h. 
\[ T(dx) \ket{x} = \ket{x+dx} \]
Für 1 Teilchen gilt mit Quantenmechanik in 3 Dimensionen: 3er Impuls, Operator 
$\hat p$, Eigenzustand $\ket{\vec{p}}$ und die Norm 
\[ \braket{\vec{p}}{\vec{p'}} = \delta^{(3)} \left( \vec{p} - \vec{p'} \right) 
\]
\\
\textbf{Relativistisch, viele Teilchentypen:}\\
\begin{itemize}
\item 4er Impuls Operator $p^{\mu}$ 
mit $p^{\mu} =$ Hamiltonoperator = Energie, \\
$\vec{p} :=$ räumlicher Impuls und \\
$p^0, \vec{p} :=$ Gesamtenergie/Impuls aller Teilchen
\item das Teilchensystem A,B,C,... wird charakterisiert durch: \begin{itemize}
\item Spin $S_{A,B,C,...}$ (hier nicht weiter definiert)
\item Ruhemasse m$_{A,B,C}$
\end{itemize}
\item Impuls-Eigenzustand für Teilchen $\ket{A,p}$ sind durch 
\[ p^{\mu} \ket{A,p} = \hat{p}^{\mu} \ket{A,p} \] und 
\[ p^2 = m_A^2 \to p^0 = \sqrt{\vec{p}^2 + m_a^2} := E_A(\vec{(p)} \] 
festgelegt.
\end{itemize}
\textbf{Normierung (relativ):} \\
\[ \braket{A,p}{A,p'}  \coloneqq  (2 \pi)^3 2 p^0 
\delta^{(3)}(\vec{p}-\vec{p'})\, \delta_{A,A'} mit p^0 =E_A(\vec{p}) \] \\ 
Der Grund hierfür ist die lorentzinvarianz: \[N(p,p') \coloneqq E_A(\vec{p}) 
\delta^{(3})(\vec{p}-\vec{p'}) \, \Rightarrow N(\Lambda p,\Lambda p') = N(p,p') 
\] 
Für die Wellenfunktion der Teilchenart $A$ mit dem Impuls $p$ schreibt man: 
$\ket{A,p}$ \\
Für einen nicht wohl-definierten Impuls folgt für das Wellenpaket: \\
\[\ket{A,f} = \int d \tilde{p} \ket{A,p} \braket{A,p}{A,f}\] \\
hierbei wurde die Vollständigkeit verwendet: 
\[\int d \tilde{p} \ket{A,p} \bra{A,p} = 1 \] 
mit \\
\[d \tilde{p} = \frac{d^3 p}{(2 \pi)^3 2 E(\vec{p})} \; \; \text{Maß} \left(= 
\frac{d^4 p}{(2 \pi)^3} \delta(p^2-m_A^2) \right)\] \\
Verteilung: $ \braket{A,p}{A,f} = f_A(\vec{p})$ mit $\ket{A,f} = \int d 
\tilde{p} f_A (\vec{p}) \ket{A,p}$ \\
Es gilt die Normierung: $\braket{A,f}{A,f}=1$ \\ \\

$\Rightarrow$ lokalisiert in Impuls mit Unschärfe, parametrisiert durch 
$f(\vec{p})$ \\
($f(\vec{p})$ ist für Teilchen, die nicht alle den gleichen Impuls haben, eine 
Wahrscheinlichkeitsverteilung.)

\section{S,T-Matrix}

Streuung (wildes Bild hier) \\
\textbf{Vorraussetzung:}

\begin{itemize}
\item falls Teilchen weit entfernt $\rightarrow$ WW vernachlässigbar
\item einlaufender Zustand präpariert bei $ t \rightarrow -\infty$
\item $\ket{i} = \ket{(A_1,p_1)(A_2,p_2)}$ zwei entfernte quasi-freie Teilchen 
(auch für $m \rightarrow n$ Streuung), $i \mathrel{\hat=}$ \glqq in $\!$\grqq
\end{itemize}
\textbf{Frage:} Wahrscheinlichkeitsamplitude dafür, bei $t \to \infty$ den 
Zustand $\ket{f}=\ket{p'_1 \dots p'_n}$ zu messen  \\
$S_{fi} \coloneqq \brazket{f}{S}{i} \rightarrow$ wichtige Größe, folgt aus 
Rechnung in QM/QFT\\
$S_{fi}$ ist die sogenannte S-Matrix ("'Scattering-Matrix"') \\ \\

4er-Impuls-Erhaltung $\Rightarrow$ Einführung der T-Matrix: 

\[ S_{fi} = \delta_{fi} + i (2\pi)^4 \delta^{(4)}\underbrace{(\sum_i p_i - 
\sum_f p'_f )}_{\tiny{\begin{matrix}\textit{Impulserhaltung aus} \\ 
\textit{Translationsinvarianz}\end{matrix}} } T_{fi} \] \\
$ \longrightarrow \delta_{fi}=0$ für $\ket{i} \neq \ket{f}$, beschreibt den 
Fall ohne Streuung\\
$ \longrightarrow T_{fi}$ beschreibt die nicht-triviale Streuung

\section{Wirkungsquerschnitt}

(hier 3 wilde Bilder) \\ \\
\textbf{einfache Definition des Wirkungsquerschnitts} $\sigma$ (WQ) \\
\\
$\frac{d \sigma}{d \Omega} = \frac{N \textit{Streuung (in } d\Omega 
\textit{)}/t}{n/t} d \Omega$ mit $m= \frac{\textit{Teilchen}}{\textit{Fläche}}$ 
für den einlaufenden Strahl. (2. dimensionale Verteilung) \\
\\
n$= \varrho_A l_A \varrho_B l_B A_{eff}$ mit: \begin{itemize}
\item $\varrho_{A,B}: $Teilchendichten in Strahl A,B
\item $l_{A,B}: $ Länge des Strahl A,B
\item $A_{eff}: $ effektive Fläche (Überlapp der beiden Strahlen)
\end{itemize}
weitere Annahmen: \begin{itemize}
\item $N_{Str}/t =$ konstant
\item $n/t =$ konstant
\end{itemize}
$\Rightarrow \sigma = \frac{N_{Streu}}{n} \to$ \framebox{$ N_{Streu} = \sigma 
\cdot n $} \\
\\
Frage: kann $\sigma$ strahlunabhängig berechnet werden?\\
Antwort: ja, unter bestimmten Annahmen \\
\\
\textbf{jetzt: Quantenmechanische Beschreibung}\\
Strahl 1: statistisches Ensemble von Teilchen A. Diese sind durch je ein 
Wellenpaket beschrieben. Gleiches gilt für Strahl 2. \\
jetzt: statistische Verteilung in xy-Ebene. \\
$\mu_1(\vec{x_1}), \mu_2(\vec{x_2})$, mit $\mu_{1,2}: $ 
Wahrscheinlichkeit/Fläche. \\
\begin{center}
es gilt: \framebox{$N_i \mu_i = \varrho_i l_i$} \\
\end{center}
Annahme: \begin{itemize}
\item alle Teilchen in Strahl 1/2 streuen unabhängig voneinander
\item es findet keine Mehrfachstreuung statt
\end{itemize}
dann gilt: Anfangszustand des Systems wird durch die Dichtematrix beschrieben: 
\\
\[ \int d\vec{x_1} d\vec{x_2} \mu_1(\vec{x_1} \mu_2(\vec{x_2} 
\ket{i_{\vec{x_1},\vec{x_2}}} \bra{i_{\vec{x_1},\vec{x_2}}}\] \\
hier steht $\ket{i_{\vec{x_1}}}$ für das Wellenpaket, welches um $\vec{x_1}$ 
bzw. $\vec{x_2}$ zentriert ist. \\
Für die zentrierten Wellenpakte gilt: 
\begin{align*}
\vec{x_1}: \ket{i_{\vec{x_1}}} = \int d\tilde{p}\, f(p) \mathrm{e}^{-i\vec{p} 
\vec{x_1}} \ket{p} \\
\vec{x_2}: \ket{i_{\vec{x_2}}} = \int d\tilde{q}\, g(q) \mathrm{e}^{-i\vec{q} 
\vec{x_2}} \ket{q}
\end{align*}
wobei \begin{align*}
\vec{x_1}: \ket{i_{\vec{x_1}}} = \int d\tilde{p}\, f(p) \ket{p} \textit{um 0}\\
\vec{x_2}: \ket{i_{\vec{x_2}}} = \int d\tilde{q}\, g(q) \ket{q} \textit{um 
$\vec{x}$}
\end{align*} zentriert ist. \\
\[\Rightarrow \ket{i_{\vec{x_1},\vec{x_2}}} = \int d\tilde{p} d\tilde{q} f(p) 
g(q) \mathrm{e}^{-i\left(\vec{p} \vec{x_1} + \vec{q} \vec{x_2} \right)} \ket{p 
q} \]
Für $\bra{i_{\vec{x_1},\vec{x_2}}}$ gilt dies analog. $\mathrm{e}^{-i[...]} \to 
\mathrm{e}^{i[...]}$ \\
jetzt: Wahrscheinlichkeit für Streuung in den Endzustand $\ket{p_1, ... , 
p_n}$\\
\[W_{if} = {\left| \brazket{f}{s}{i} \right|}^2 = ... = \int d\vec{x_1} 
d\vec{x_2} \mu_1(\vec{x_1}) \mu_2(\vec{x_2}) 
\brazket{i_{\vec{x_1},\vec{x_2}}}{s^{\dagger}}{p_1', ... , p_n'} \brazket{p_1', 
... , p_n'}{s}{i_{\vec{x_1},\vec{x_2}}} \]

Wildes Bild erscheint.\\
Ein weiterer wichtiger Parameter, den wir nur einführen, ist der 
Impaktparameter $\vec{b} = \vec{x_1} - \vec{x_2}$. \\

\textbf{Annahme 1:} Nur der Impaktparameter $\vec{b}$ soll relevant sein. \\
$\Rightarrow$ Translationsinvarianz \\
Ergebnis: $W_{if} \sim \int d\vec{x_1} \mu_1(\vec{x_1}) 
\mu_2(\vec{x_2-\vec{b}}) = \mu(\vec{b}) = 
\frac{\textit{Wahrscheinlichkeit}}{\textit{Fläche}} $ des Impaktparameters 
$\vec{b}$. \\

\textbf{Annahme 2:} Ww. hat endliche Reichweite, über gesamte Reichweite $\mu 
(\vec{b}) = \mu (0) = $ konstant. \\
\[ W_{if} = \mu(0) \int d\vec{p} {\left| f(p) \right|}^2 d\vec{q} {\left| g(q) 
\right|}^2 A_1 \frac{1}{4p^0 q^0 \Delta V} \]
mit \[ \Delta V = \left| V_p - V_q \right|, V_{rel} = \frac{\left| \vec{p} 
\right|}{E}\] 
aus $E= \gamma m ,\ \vec{p} = \gamma m \vec{v}\ (c=1)$ \\
mit \[ A = {(2\pi)}^4 \delta^{(4)} \left(p+q - \sum_f p_f) 
T^{\dagger}_{\bar{p}\bar{q} \to \left\lbrace p_f \right\rbrace} T_{pq \to 
\left\lbrace p_f \right\rbrace} \right)\] 
des Weiteren gilt analog: 
\[ \bra{i_{\vec{x_1} \vec{x_2}}} = \int d\tilde{\bar{p}} d\tilde{\bar{q}} 
\dotsc \] 
\textbf{Annahme 3:} Wellenpakete haben gut definierten (=relativ scharfen) 
Impuls p bzw. q in z-Richtung. $p= \left( p_0, 0, 0, p_z \right), q = \left( 
q^0, 0, 0, q_z \right)$ \\
$\Rightarrow$ über gesamtes Wellenpaket sind $\Delta V$ und A $\sim$ konstant. 
\\
\[ \int d\tilde{p} p_0 \approx \frac{1}{p_0} \int d\tilde{p} f(p)\ \textit{, 
analog für q.} \] 
\[ W_{if} = \frac{\mu(0)}{4 \sqrt{{(p-q)}^2 - m_A^2 m_B^2}} {(2 \pi)}^4 
\delta^{(4)} (p+q - \sum_f{p_f}) \left| T_{pq \to \left\lbrace p_f 
\right\rbrace} \right| \]
mit $m_{A,B}$: Massen der Teilchen in Strahl 1,2 \\
$\Rightarrow$ Ergebnis: 
\begin{itemize}
\item Strahlabhängige Größe $\mu(0)$ 
\item Strahlunabhängige, in Theorie berechnete Größe 
\end{itemize} 
Analog im Impulsraum existieren f Endzustände $d\tilde{p_1}' \dotsc 
d\tilde{p_n}' \ket{p_1' \dotsc {p_n}'} \bra{p_1' \dotsc p_n'}$ \\
\\
\textbf{Wildes Bild erscheint}\\
\\
$\rightarrow d\sigma (p_1' \dotsc p_n') = \frac{{\left| T_{pq \to p_1' \dotsc 
p_n'} \right| }^2 }{\sqrt{4 {(p-q)}^2 - m_A^2 m_B^2}} \underbrace{{(2 \pi)}^4 
\delta^{(4)} (p+q - \sum_f{p_f}) d\tilde{p_1}' \dotsc d\tilde{p_n}'}_{=d\Phi_n 
:= \textit{Phasenraum}} $ \\
\\
$\Rightarrow dN_{Streuung} = N_{Strahl1} \cdot N_{Strahl2} \cdot \mu(0) 
d\sigma$ \\
\\
Es wird nun die Definition des Wechselwirkungsquerschnitts sowie die Formel von 
d$\sigma$ verbunden. \\
\begin{align*}
N_{Streu} \left[ d\Omega \right] &= W_{if} \left[ d\Omega \right] N_1^{A_{eff}} 
N_2^{A_{eff}} = \mu\left( 0 \right) d\sigma N_1^{A_{eff}} N_2^{A_{eff}} \\
&= d\sigma \underbrace{\int_{A_eff} d\vec{x_1} \mu_1 \left( \vec{x_1} \right) 
\left( \vec{x_2} \right) N_1^{A_{eff}} N_2^{A_{eff}}}_{\# Teilchen/Fläche}
\end{align*}
mit \[ \mu_1 = \mu_2 =\ \mathit{konstant} \] 
und
\[ N_{Streu} \left[ d\sigma \right] = d\sigma \cdot n \hspace{1cm} d\sigma 
\coloneqq \mathit{ganzes\ Winkelelement} \]
Daraus folgt: \[ N_{Streu} = \sigma \cdot n \hspace{1cm} \]
mit \[ n = \frac{\# Teilchen}{Fläche} \]

\section{Ergebnisse für 2 $\to$ 2}

\textbf{Streuung:} \\
\\
\textbf{(wildes Bild)}\\

\[ d\sigma = \frac{|T_{if}|^2}{s \sqrt{(p_1 p_2)^2-m_1^2 m_2^2}} d \Phi_2, 
\quad   d \Phi_2 = (2 \pi)^4 \delta^{(4)}(p_1+p_2-p'_1-p'_2) d \tilde{p}'_1 d 
\tilde{p}'_2 \]
Ein typisches System ist das Schwerpunktsystem, für das gilt:
\[ \sum_i \vec{p_i} = 0 \]
\[ p_1+p_2 = \underbrace{(\sqrt{S}, \vec{0})}_Q \] 
mit $\sqrt{S}$ als Energie im Schwerpunktsystem \\
nach Berechnungen auf dem Übungsblatt 1: 
\[ d \sigma = \frac{|T_{if}|^2}{2 \sqrt{\lambda_i}} \frac{1}{4 (2 \pi)^2} 
\frac{\sqrt{\lambda_f}}{2s} d \Omega_1 \]
für $2 \rightarrow 2$ Streuung \\
\[ \lambda(a,b,c) \coloneqq a^2+b^2+c^2-2ab-2ac-2bc \]
mit 
\[ \lambda_i = \lambda(s, m_1^2, m_2^2) \hspace{0.25cm} \textrm{ und } 
\hspace{0.25cm} \lambda_f = \lambda(s, {m'_1}^2, {m'_2}^2) \]
$ ^{(*)} d \sigma = \frac{1}{64 \pi^2 s} |T_{if}|^2 \d \Omega_1 
\frac{\sqrt{\lambda_f}}{\sqrt{\lambda_i}} \quad $  impliziert, dass 
\hspace{1cm}  \parbox{3cm}{ \vspace{2em} $p'_2= p_1+ p_2 -p'_1$ \\ 
$|\vec{p'_1}|=|\vec{p'_2}|= \frac{\sqrt{\lambda_f}}{2 \sqrt{s}}$} \\

\textbf{Sonderfall:} $m_1 = m_2 = m_1' = m_2'$, $\lambda_f = \lambda_i$ 
Daraus folgt:
\[ \frac{d\sigma}{d\Omega} = \frac{1}{64 \pi^2 s} {\left| T_{fi} \right|}^2 \]
\textbf{Anmerkungen:} $d\sigma \sim \sqrt{\lambda_f}$. \\
z.B.: \begin{itemize}
	\item $m_1' = m_2' = 0 \Rightarrow s^2$ \\
	$\left| p_1' \right| = \left| p_2' \right| = \frac{\sqrt{s}}{2} \rightarrow 
	\textit{Maximum} $
	\item $m_1' + m_2' \approx \sqrt{s} \rightarrow \left| \vec{p_1}' \right| 
	\approx \left| \vec{p_2}' \right| \approx 0$ \\
	an dieser "'Schwelle"' werden die Teilchen quasi in Ruhe erzeugt
\end{itemize}
Im Allgemeinen gilt: Wenn $m_1 + m_2$ größer werden, wird $\left| \vec{p_1}' 
\right|$ kleiner bei gleichen s. \\
\\
\textbf{Einheiten:} 
\[ \left[ d\sigma \right] = \mathit{Fläche} \sim \frac{1}{{\left[ E \right]}^2} 
\]
Für 2 $\to$ 2 folgt aus $ ^{(*)}$, dass $T_{fi}$ dimensionslos ist. \\
\\
Wir betrachten $ ^{(*)}$ im Schwerpunktsystem (oder allgemein). $\vec{p_{1,2}}$ 
liegt fest. \\
$d\tilde{p_1}', d\tilde{p_2}' \to$ 2x3 Variablen und aus $\delta^{(4)}$ folgt, 
dass es 4 Bedingungen sind. \\
Es ergeben sich: \\
$6-4 =2 $ Freiheitsgrade und daraus $d\Omega \to$ Winkelverteilungen \\
\\
z.B.: 
\[ \sqrt{s} >>\underbrace{ m_1, m_2, m_1', m_2'}_{\approx 0} \Rightarrow 
\frac{d\sigma}{d\Omega} \sim \frac{1}{s}\] (ist der Hochenergielimes [z.B. für 
t= konstant])